\documentclass{article}
\usepackage[utf8]{inputenc}
\usepackage{amsmath}
\usepackage[
  left=3cm,
  right=2cm,
  top=2.5cm,
  bottom=2cm,
]{geometry}

\begin{document}
Tatiana Nazarova, 6328375; Philip Spahn, 6043168
\section*{Dokumentation: Praxisaufgabe 1}
Der normal gedruckte Text ist das Texttrnskript, der mittig gedruckte und kursic gedruckte Text beschreibt das was gerade auf der Webseite passiert. Die fettgedruckten Überschriften zeigen an auf welcher Seite man sich gerade befindet.

\subsection*{Drehbuch}
\begin{center}
\textbf{--STARTSEITE--}
\end{center}
Bei der Startseite wurde größtenteils mit dem Prinzipip der gemeinsamen Region, des gemeinsamen Schicksals und dem prinzip der Ähnlichkeit der Farbe und Form gearbeitet. Dies sieht man zum Einen an den Buchkartenbuttons...
\begin{center}
\textit{-- MAUS KREIST UM DIE BUCHKARTENBUTTONS --}
\end{center}
und zum Anderen an den Buchkarten.
\begin{center}
\textit{--MAUS KREIST UM DIE BUCHKARTEN--}
\end{center}
Die Buchkartenbuttons sind alle in kleine Rechtecke unterteilt und haben eine andere Farbe als der Hintergrund, somit werden sie durch das Prinzip der Farbe von dem Hintergrund und durch die andere Form von den Buchkarten, abetrennt. Fährt man mit der Maus über einen Buchkartenbutton...
\begin{center}
\textit{--MAUS FÄHRT ÜBER BUCHKARTENBUTTON--}
\end{center}
So ändert sich die Farbe des Buttons und unterscheidet sich von den anderen. Somit wird das Prinzip der gemeinsamen Farbe eingesetzt, um den ausgewählten Button von den nicht ausgewählten Buttons abzugrenzen.\\
\begin{center}
\textit{-MAUS FÄHRT ÜBER DEN TITEL IN DEN BUCHKARTEN-}
\end{center}
Die Titel sind in den Buchkarten durch einen Strich von der Beschreibung des Buches getrennt, somit ist in den Buchkarten das Prinzip der gemeinsamen Region ebenfalls umgesetzt. Die Buchkarten haben ebenfalls wie die Buttons eine andere Farbe wie der Hintergrund, um davon abgegrenzt zu werden. Sie haben jedoch die gleiche Farbe wie die Buckartenbuttons, um den Funktinaltätszusammenhang klar zu machen. Da wenn man auf die Buttons klickt, das entsprechende Buch automatisch eingeblendet wird.
\begin{center}
\textit{-MAUS KLICKT AUF MEHRERE BUCHBUTTONS UND FÄHRT DANACH ZUERST ÜER DAS BILD IN DER BUCHKARTE UND DANN ÜER DEN BECSHREIBUNGSTEXT UND DANACH ZUR ÜBERSCHRIFT-}
\end{center}
Alle Buchkarten sind gleich aufgebaut, damit man mit Hilfe des Designprinzips der Ähnlichkeit merkt, dass sie zusammen gehören. Bei allen Buchkarten ist das Bild links angeordnet, der Becshreibungstext rechts daneben und der Titel abgetrennt dadrüber.
\begin{center}
\textit{-MAUS SCROLLT DIE BUCHKARTEN MEHRMALS DURCH-}
\end{center}
Wie man sieht bewegen sich alle Buchkarten, wenn man die Scrollbar benutzt dadurch sind die Buchkarten durch das PRinzip des gemeinsamen Schicksals miteinander verbunden, da sie sich alle gemeinsam in eine Richtung bewegen.\\
\\
Klickt man nun auf den Navigationsbutton mit der Inschrift 'Katalog' auf der linken Seite, so kommt man zum Katalog.
\begin{center}
\textit{-MAUS KLICKT AUF DEN BUTTON 'KATALOG'-}
\end{center}
\begin{center}
\textbf{--KATALOG--}
\end{center}
Ähnlich wie bei der Startseite wurde auch hier größtenteils mit dem Prinzip der gemeinsamen Region, der gemeinsamen Farbe und des gemeinsamen Schicksals gearbeitet.\\
Generell ist der Aufbau der Seite ähnlich zu dem der Hauptseite, damit der User sich nicht umgewöhnen muss. Die Buchkartenbuttons befinden sich auf der linken und die Buchkarten auf der rechten Seite.
\begin{center}
\textit{-MAUS GEHT ÜBER DIE BUCHKARTENBUTTONS LINKS UND DANACH ÜBER DIE BUCHKARTEN RECHTS-}
\end{center}
Ein Unterschied zu Startseite ist, dass nun die Buchkartenbuttons in verschiedenfarbenen Grüppchen durch Überschriften getrennt aufgeteilt sind. Hier wird das Prinzip der gemeinsamen Region dafür eingesetzt die Buchkartenbuttons nach Genre zu ordnen, dazu dient auch hier das Prinzip der gemeinsamen Farbe, denn jedem Genre wurde eine Farbe zugeordnet. Die zugeordneten Farben finden sich auch in den Titeln der Bücher um eine Verbindung durch das Prinzip der Ähnlichkeit durch Farbe herzustellen und zu verdeutlichen, dass die Buchkarten zu den Buchkartenbuttons der jeweiligen Farbe gehören.
\begin{center}
\textit{-MAUS UMKREIST JEDE GRUPPE DER BUCKARTENBUTTONS UND WECHSET DANACH ZWISCHEN DEN BUCHKARTENBUTTONS UND DEN JEWEILIGEN BUCHKARTEN HIN UND HER}
\end{center}
Bei den Buchkarten wurden die gleichen Designprinzipien umgesetzt wie bereits in der Startseite beschrieben.\\
\\
Nun kommen wir zur Funktionalität der Suchmaske:
\begin{center}
\textit{-MAUS KLICKT DEN FILTER-BUTTON-}
\end{center}
Klickt man auf den 'Filter'-Button so erscheint eine Suchmaske mit mehreren Feldern. Man muss in die Felder die links neben dem Suchfeld geforderte Information, zum Beispiel Titel, ISBN oder Autor, etc. eingeben und danach auf anwenden klicken. Dies wird hier an einem Beispiel demonstriert.
\begin{center}
\textit{-MAUS BEWEGT SICH ZUM TEXTFELD 'TITEL', KLICKT AUF DAS FELD UND GIBT 'Bored of the Rings' EIN UND KLICKT DANACH AUF 'ANWENDEN'. ALLE BUCKARTEN AUSSER DEM MIT DEM BUCH 'BORED OF THE RINGS' WERDEN NICHT MEHR ANGEZEIGT.}
\end{center}
Gibt man einen falschen Wert ein, so werden keine Bücher angezeigt.
\begin{center}
\textit{-MAUS LÖSCHT DEN EINGEGEBENEN TITEL, GEHT AUF SUCHFELD AUTOR, KLICKT DARAUF UND GIBT 'BLABLA' EIN. ALLE BUCHKARTEN VERSCHWINDEN.}
\end{center}
Wird nach mehreren Kriterien gefiltert, so wird ein Buch nur dann angezeigt, wenn beide korrekt sind.
\begin{center}
\textit{-MAUS GEHT ZUM FELD ISBN UND GIBT DIE KORREKTE ISBN VON 'BORED OF THE RINGS' EIN. ES WERDEN IMMERNOCH KEINE BÜCER ANGEZEIGT.}
\end{center}
Gibt man einen Wert in die Preisspanne ein, so werden einem alle Bücher einschließlich des Grenzwerts angezeigt.
\begin{center}
\textit{-MAUS LÖSCHT ALLE EINGEGEBENEN DATEN UND GIBT ALS OBERE GRENZE 12 EIN UND DRÜCKT AUF ANWENDEN. ALLE BÜCHER INNERHALB DER PREISSPANNE WERDEN ANGEZEIGT.}
\end{center}
Drückt man wenn man mit der Suche fertig ist nochmals auf den Button 'Filter' so wird die Anssicht zurückgesetzt.
\begin{center}
\textit{-MAUS DRÜCKT AUF DEN BUTTON 'FILTER' UND ALLE BÜCHER WERDEN ERNEUT ANGEZEIGT.}
\end{center}
Drückt man nun bei den linken Navigationsbuttons auf den Button 'Gewinnspiel' gelangt man zum Gewinnspiel.
\begin{center}
\textit{-MAUS BEWEGT SICH ZUM BUTTON GEWINNSPIEL UND BETÄTIGT IHN-}
\end{center}
\begin{center}
\textbf{--GEWINNSPIEL--}
\end{center}
Auch hier wurden dieselben Designprinzipien verwendet, wie bei der Startseite und beim Katalog auch. Das Fragenfeld des Gewinnspiels ist durch das Prinzip der gemeinsamen Region und das Prinzip der Farbe, durch das Rechteck von dem Hintergrund und dem restlichen Text abgetrennt. 
\begin{center}
\textit{-MAUS KREIST UM FRAGEFELD-}
\end{center}
Beantwortet man die Fragen, so bekommt man als Reaktion darauf die Bewertung und einen Smiley in Form unseres Maskotchens Wurmi. Es kommen immer unterschiedliche Smileys, je nachdem wie die Frage beantwortet wird.
\begin{center}
\textit{-MAUS KLICKT AUF DIE ANTWORTEN, ZUERST WIRD DIE ERSTE FRAGE FALSCH BEANTWORTET, ES ERSCHEINT EIN TRAURIGES EMOTICON. DANACH WIRD DIE FRAGE RICHTIG BEANTWORTET UND ES ERSCHEINT EIN FRÖHLICHES EMOTICON. DANACH WIRD DAS SPIEL GEWONNEN UND ES ERSCHEINT EIN EMOTICON MIT BRILLE.}
\end{center}
Dieses Emoticon ist auch unser Alleinstellungsmerkmal, da es das Maskottchen unserer Buchhandlund darstellt und sich auch im Logo und im Namen unserer Buchhandlung widerfindet. 
\begin{center}
\textit{-MAUS ZEIGT AUF EMOTICON, DANACH AUF DAS LOGO UND DANACH AUF DEN TITEL-}
\end{center}
Generell ist noch anzusprechen, dass der Button 'Dark-Theme' und 'Light-Theme' die Fraben lediglich invertiert, die besprochenen Designprinzipien bezüglich der Farbe bleiben also erhalten.
\begin{center}
\textit{-MAUS KLICKT AUF DEN 'DARK-THEME' BUTTON, DAS THEME INVERTIERT SICH. DANACH KLICKT SIE AUF DEN 'LIGHT-THEME' BUTTON UND ES INVERTIERT SICH ZURÜCK.-}
\end{center}
Die Farben unserer Webseite wurden so gewählt, das Menschen mit einer Grün und Blausehschwäche die Seite sehen können, außerdem wurden warme Rot-Töne gewählt, damit der User sich wohler fühlt.

\end{document}